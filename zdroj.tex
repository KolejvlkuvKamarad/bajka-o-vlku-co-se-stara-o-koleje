% Bajka o vlku, co se stará o koleje
% Autor: kolejvlkův kamarád
% 
% Tento soubor je vydán pod licencí CC0 1.0. (Text licece zde: https://creativecommons.org/publicdomain/zero/1.0/). Já, autor, kolejvlkův kamarád, se tudíž vzdávám autorských práv k tomuto souboru.
% Pokud chcete zveřejnit úpravu tohoto souboru, doporučuji náležitě upravit tento copyright notice a označit se jako autora změn. Ale nemůžu vás nutit; je to jenom doporučení, aby nebyl bordel.

\documentclass[a5paper, twoside,12pt]{book}
\usepackage[utf8]{inputenc}
\usepackage[IL2]{fontenc}
\usepackage[czech]{babel}
\usepackage{geometry}
\usepackage{verse}
\usepackage{qrcode}
\usepackage{svg}
\usepackage[symbol]{footmisc}
\usepackage[normalem]{ulem}
\renewcommand{\thefootnote}{\fnsymbol{footnote}}
\graphicspath{{graphics/}}
\title{Bajka o~vlku, co se staral o~koleje}
\begin{document}

\begin{titlepage}
    \newgeometry{top=2cm,bottom=2cm,left=2cm,right=2cm}
    \begin{center}
        {\Large \textsc{Kolejvlkův kamarád}}

        \vspace{\stretch{0.382}}
        {\LARGE Bajka o~vlku, co se staral o~koleje}\\
        \vspace{\stretch{0.618}}
        {\large první vydání} \hfill {\large březen 2022}
        \end{center}
\end{titlepage}

\newgeometry{top=2cm,bottom=2cm,left=2cm,right=2cm}
\thispagestyle{empty}
\vspace*{\fill}
\emph{Budiž tato \uv{kniha} udělá někomu radost\footnote[2]{a pokud ne radost, tak alespoň pravý a nefalšovaný WTF pocit, který může nastat pouze na VUT v~Brně}, v~tomto nebo příštích zkouškových období nebo i pokud jen někdo potřebuje povzbudit.}\\
\vspace*{\fill}

\restoregeometry
\setcounter{page}{1}
Je pozdní večer, první máj, na Purkyňových kolejích. V~oslňujícím svitu fluorescenčního osvětlení kuchyňky stojí dva studenti a hluboce bloumájí před otevřenou mikrovlnkou.

\uv{Pojďme to, \emph{škyt} znova udělat.}

\uv{Ale... a... Ale není teď tady nějaká nová přísná provozní? Myslím, že to je možná i kolejchlap.}

Jeho kamarád si odbleje do blízkého umyvadla, nebo alespoň si myslí, že tam jeho obsah žaludku směřuje.

\begin{verse}
Kolejbába, kolejchlap.\\
Přišli zde jen,\\
aby nás mohli srát.\\
\ \\
Každý ráno, každý večer.\\
`Serte jen do záchodu!'\\
Já bych brečel.\\
\ \\
Není se čeho bát.\\
Tohle jsou přeci Purkyně.\\
Pojďme do mikrovlnky srát.\\
\end{verse}

Umělecký výtvar, na který přišli poslední tři aktuálně aktivní mozkové buňky studenta Fakulty výtvarných umění -- jejich houževnatost způsobena četnými projekty z~předmětů kreativního psaní -- se dotkla jeho kumpána, který ihned a bez vyzvání začal sundávat své kalhoty.

Křik. \uv{Ti parchanti zkurvení to udělali znova!} zaznívá z~kuchyňky v~sedm ráno veleznámý hlas místní uklízečky. Mnoho ze studenstva je nyní probuzeno -- ti v~prvním ročníku zhrozeni; pro ty ve třetím je to již prostá realita věci.

Zdrženlivější ze dvou včerejších studentů sice měl pravdu, že nový provozní je kolejchlap. Kolejchlapem obecným však bohužel není. Pocházející ze světa barvitých neonových zvířat, nadprůměrné velikosti osobního rozpočtu pro umělecké zakázky a jisté fascinace pro webovou stránku pojmenovanou po potravinovém aditivu glutamátu sodnému, nový provozní je kolejvlk.

Kolejvlk vidí zhrozený pohled jeho podlahy vytírající kolegyně. \uv{To nebyl vtip?} ptá se. Kolegyně odpoví, \uv{Však jste tady studoval, ne?}

Kolejvlk, těžce nesoucí vlnu reality, že všechny báchorky, které slýchal v~době svého studia, byly pravdivé, se zvedá ze svého křesla na vrátnici a začíná svůj pochod k~oné kuchyňce.

\uv{Kdo se vysral do mikrovlnky?} ptá se hlasitě a zřetelně. Žádná odpověď se ale nedostaví. Kolejvlk dotaz několikrát opakuje, opět bezvýsledně. 

V~tento moment se s~kocovinou způsobenou bolestí hlavy -- stále bez kalhot -- probouzí dva viníci. Zdrženlivější z~nich cítí trochu viny, ale oba ví, že pokud se nikdo nepřizná, tak se tato událost pouze zaeviduje jako jedna z~mnoha kuriozit života na Purkyňových kolejích. \uv{Však, tak to bylo vždy a proč by tomu mělo být dnes jinak?} řekne si jeden z~nich.

Kolejvlk si těžce a hluboce povzdechne. Po troše přemýšlení a nabírání odvahy uřkne slova: \uv{UwU! Gdo tady udělal ne-ne do miqwovnqny? OwO?}

Slova rezonují v~okruhu dvanácti pokojů na všechny světové strany. Kdo si ještě nebyl vědom nebo ignoroval dění v~Purkyňské kuchyňce, nyní dává svůj plný pozor. Oba včerejší studenti se na sebe dívají zděšeným pohledem.

\uv{PwOwOsím stwafte se na vrátnici. Jinak na žádňoučkém patříčkUwU nebude žádná miqwovnqna. ÒwÓ} naposled pronesl kolejvlk a vrátil se na svůj trůn na vrátnici.

Uběhl celý jeden den. Uklízečka, nyní již po své první návštěvě profesionálního psychologa, se mentálně připravuje vstoupit do kuchyňky. Když vstoupí, tak se ji prvně uleví, poté je překvapena. Na místě, kde včera byla odebrána mikrovlnná trouba se nachází zbrusu nová.

\uv{Dobrý den, paní Mirko.} pozdraví ji hologram promítaný z~mikrovlnky. Paní Mirka náležitě odpoví. Překvapena, po douklizení bloku se vrací na vrátnici ke kolejvlkovi a vypráví mu o~zjevení v~dobře známé kuchyňce. Kolejvlk je také překvapen, ale je rád, že studenti Purkyňových kolejích konečně pro jednou napravili své chyby.

Kolejvlk, spokojen, se podívá na svůj telefon a pokračuje procházet obsah webu Živě.cz, kde vidí titulek, \uv{Na Consumer Electronics Show v~Las Vegas byl zcizen multimilionový prototyp mikrovlnné trouby. Pachatelé umístili výkaly do zbylých prezentačních modelů.}

\newpage

\begin{center}
    Tahle stránka je prázdná. Nemáme dost textu a z důvodu stylu vazby téhle knihy tato stránka musí zde být. Tak si třeba tady nakresli kolejvlka. Nebo klidně třeba kolejlva a jeho milence menzatigra. UwU. Je to na tobě. 
\end{center}

\newgeometry{top=2cm,bottom=2cm,left=2cm,right=2cm}
\thispagestyle{empty}
\vspace*{\fill}
\begin{center}
    Bajka o~vlku, co se staral o~koleje\\
    \vspace{\baselineskip}
    první vydání, dotisk\\
    původně vydáno v~\sout{lednu}~březnu 2022\\
    \vspace{\baselineskip}
    napsal a vysázel \uv{kolejvlkův kamarád}\\
    \vspace{\baselineskip}
    Obrázek packy na obálce je dostupný jako volné dílo od autora kattekrab z~webové stránky \texttt{https://openclipart.org/detail/142447/paw-print}.\\
    \vspace{\baselineskip}
    Žádná práva vyhrazena.\\
    \emph{Bajka o vlku, co se staral o koleje} od autora \uv{kolejvlkův kamarád} je výdána pod licencí CC0 1.0 Universal. Pro zobrazení plného znění licence navštivte webovou stránku \texttt{http://creativecommons.org/publicdomain/zero/1.0}.\\
    Dílo můžete jakkoliv užít dle vlastního uvážení bez jakýkoliv omezení.\\
    \includesvg{cc-zero.svg}
\end{center}
\vspace*{\fill}
\begin{center}
zdrojové soubory (text, \LaTeX{} a PDF) lze získat odsud\\
\texttt{https://github.com/KolejvlkuvKamarad/\\bajka-o-vlku-co-se-stara-o-koleje}\\
\qrcode[height=1in]{https://github.com/KolejvlkuvKamarad/bajka-o-vlku-co-se-stara-o-koleje}
\end{center}
\newpage

\thispagestyle{empty}
\vspace*{\fill}
\begin{center}
    \includesvg[height=3cm]{Paw-print.svg}
\end{center}
\vspace*{\fill}
\end{document}
